%**************************************%
%* Generated from MathBook XML source *%
%*    on 2016-03-02T17:31:20-08:00    *%
%*                                    *%
%*   http://mathbook.pugetsound.edu   *%
%*                                    *%
%**************************************%
\documentclass[10pt,]{book}
%% Load geometry package to allow page margin adjustments
\usepackage{geometry}
\geometry{letterpaper,total={5.0in,9.0in}}
%% Custom Preamble Entries, early (use latex.preamble.early)
%% Inline math delimiters, \(, \), need to be robust
%% 2016-01-31:  latexrelease.sty  supersedes  fixltx2e.sty
%% If  latexrelease.sty  exists, bugfix is in kernel
%% If not, bugfix is in  fixltx2e.sty
%% See:  https://tug.org/TUGboat/tb36-3/tb114ltnews22.pdf
%% and read "Fewer fragile commands" in distribution's  latexchanges.pdf
\IfFileExists{latexrelease.sty}{}{\usepackage{fixltx2e}}
%% Page Layout Adjustments (latex.geometry)
%% For unicode character support, use the "xelatex" executable
%% If never using xelatex, the next three lines can be removed
\usepackage{ifxetex}
\ifxetex\usepackage{xltxtra}\fi
%% Symbols, align environment, bracket-matrix
\usepackage{amsmath}
\usepackage{amssymb}
%% allow more columns to a matrix
%% can make this even bigger by overiding with  latex.preamble.late  processing option
\setcounter{MaxMatrixCols}{30}
%% Semantic Macros
%% To preserve meaning in a LaTeX file
%% Only defined here if required in this document
%% Used for inline definitions of terms
\newcommand{\terminology}[1]{\textbf{#1}}
%% Used for units and number formatting
\usepackage[per-mode=fraction]{siunitx}
\ifxetex\sisetup{math-micro=\text{µ},text-micro=µ}\fi%% Common non-SI units
\DeclareSIUnit\degreeFahrenheit{\SIUnitSymbolDegree{F}}
\DeclareSIUnit\fahrenheit{\degreeFahrenheit}
\DeclareSIUnit\pound{lb}
\DeclareSIUnit\foot{ft}
\DeclareSIUnit\inch{in}
\DeclareSIUnit\yard{yd}
\DeclareSIUnit\mile{mi}
\DeclareSIUnit\mileperhour{mph}
\DeclareSIUnit\gallon{gal}
%% Subdivision Numbering, Chapters, Sections, Subsections, etc
%% Subdivision numbers may be turned off at some level ("depth")
%% A section *always* has depth 1, contrary to us counting from the document root
%% The latex default is 3.  If a larger number is present here, then
%% removing this command may make some cross-references ambiguous
%% The precursor variable $numbering-maxlevel is checked for consistency in the common XSL file
\setcounter{secnumdepth}{3}
%% Environments with amsthm package
%% Theorem-like enviroments in "plain" style, with or without proof
\usepackage{amsthm}
\theoremstyle{plain}
%% Numbering for Theorems, Conjectures, Examples, Figures, etc
%% Controlled by  numbering.theorems.level  processing parameter
%% Always need a theorem environment to set base numbering scheme
%% even if document has no theorems (but has other environments)
\newtheorem{theorem}{Theorem}[section]
%% Only variants actually used in document appear here
%% Numbering: all theorem-like numbered consecutively
%% i.e. Corollary 4.3 follows Theorem 4.2
%% Definition-like environments, normal text
%% Numbering for definition, examples is in sync with theorems, etc
%% also for free-form exercises, not in exercise sections
\theoremstyle{definition}
\newtheorem{example}[theorem]{Example}
\newtheorem{exercise}[theorem]{Exercise}
%% Localize LaTeX supplied names (possibly none)
\renewcommand*{\chaptername}{Chapter}
%% Figures, Tables, Listings, Floats
%% The [H]ere option of the float package fixes floats in-place,
%% in deference to web usage, where floats are totally irrelevant
%% We re/define the figure, table and listing environments, if used
%%   1) New mbxfigure and/or mbxtable environments are defined with float package
%%   2) Standard LaTeX environments redefined to use new environments
%%   3) Standard LaTeX environments redefined to step theorem counter
%%   4) Counter for new enviroments is set to the theorem counter before caption
%% You can remove all this figure/table setup, to restore standard LaTeX behavior
%% HOWEVER, numbering of figures/tables AND theorems/examples/remarks, etc
%% WILL ALL de-synchronize with the numbering in the HTML version
%% You can remove the [H] argument of the \newfloat command, to allow flotation and 
%% preserve numbering, BUT the numbering may then appear "out-of-order"
\usepackage{float}
\usepackage[bf]{caption} % http://tex.stackexchange.com/questions/95631/defining-a-new-type-of-floating-environment 
\usepackage{newfloat}
% Figure environment setup so that it no longer floats
\SetupFloatingEnvironment{figure}{fileext=lof,placement={H},within=section,name=Figure}
% figures have the same number as theorems: http://tex.stackexchange.com/questions/16195/how-to-make-equations-figures-and-theorems-use-the-same-numbering-scheme 
\makeatletter
\let\c@figure\c@theorem
\makeatother
%% Raster graphics inclusion, wrapped figures in paragraphs
\usepackage{graphicx}
%% Colors for Sage boxes and author tools (red hilites)
\usepackage[usenames,dvipsnames,svgnames,table]{xcolor}
%% More flexible list management, esp. for references and exercises
%% But also for specifying labels (ie custom order) on nested lists
\usepackage{enumitem}
%% Lists of exercises in their own section, maximum depth 4
\newlist{exerciselist}{description}{4}
\setlist[exerciselist]{leftmargin=0pt,itemsep=1.0ex,topsep=1.0ex,partopsep=0pt,parsep=0pt}
%% hyperref driver does not need to be specified
\usepackage{hyperref}
%% Hyperlinking active in PDFs, all links solid and blue
\hypersetup{colorlinks=true,linkcolor=blue,citecolor=blue,filecolor=blue,urlcolor=blue}
\hypersetup{pdftitle={Mt. Hood Precaclucus}}
%% If you manually remove hyperref, leave in this next command
\providecommand\phantomsection{}
%% Graphics Preamble Entries
\usepackage{pgfplots}
%% extpfeil package for certain extensible arrows,
%% as also provided by MathJax extension of the same name
%% NB: this package loads mtools, which loads calc, which redefines
%%     \setlength, so it can be removed if it seems to be in the 
%%     way and your math does not use:
%%     
%%     \xtwoheadrightarrow, \xtwoheadleftarrow, \xmapsto, \xlongequal, \xtofrom
%%     
%%     we have had to be extra careful with variable thickness
%%     lines in tables, and so also load this package late
\usepackage{extpfeil}
%% Custom Preamble Entries, late (use latex.preamble.late)
%% Begin: Author-provided macros
%% (From  docinfo/macros  element)
%% Plus three from MBX for XML characters

\newcommand{\lt}{ < }
\newcommand{\gt}{ > }
\newcommand{\amp}{ & }
%% End: Author-provided macros
%% Title page information for book
\title{Mt. Hood Precaclucus\\
{\large Elementary Functions}}
\author{Jack Green\\
Department of Mathematics\\
Mt. Hood Community College\\
\href{mailto:jack.green@mhcc.edu}{\nolinkurl{jack.green@mhcc.edu}}
\and
Nick Chura\\
Department of Mathematics\\
Mt. Hood Community College\\
\href{mailto:nickolas.chura@mhcc.edu}{\nolinkurl{nickolas.chura@mhcc.edu}}
}
\date{March 2, 2016}
\begin{document}
\frontmatter
%% begin: half-title
\thispagestyle{empty}
{\centering
\vspace*{0.28\textheight}
{\Huge Mt. Hood Precaclucus}\\[2\baselineskip]
{\LARGE Elementary Functions}\\
}
\clearpage
%% end:   half-title
%% begin: adcard
\thispagestyle{empty}
\null%
\clearpage
%% end:   adcard
%% begin: title page
%% Inspired by Peter Wilson's "titleDB" in "titlepages" CTAN package
\thispagestyle{empty}
{\centering
\vspace*{0.14\textheight}
{\Huge Mt. Hood Precaclucus}\\[\baselineskip]
{\LARGE Elementary Functions}\\[3\baselineskip]
{\Large Jack Green}\\[0.5\baselineskip]
{\Large Mt. Hood Community College}\\[3\baselineskip]
{\Large Nick Chura}\\[0.5\baselineskip]
{\Large Mt. Hood Community College}\\[3\baselineskip]
{\Large March 2, 2016}\\}
\clearpage
%% end:   title page
%% begin: copyright-page
\thispagestyle{empty}
\vspace*{\stretch{2}}
\noindent\copyright\ 2010\textendash{}2016\quad{}Jack Green, Nick Chura\par
\vspace*{\stretch{1}}
\null\clearpage
%% end:   copyright-page
%% begin: table of contents
\setcounter{tocdepth}{1}
\renewcommand*\contentsname{Contents}
\tableofcontents
%% end:   table of contents
\mainmatter
\typeout{************************************************}
\typeout{Chapter 1 Functions}
\typeout{************************************************}
\chapter[Functions]{Functions}\label{functions}
\typeout{************************************************}
\typeout{Section 1.1 Student Learning Outcomes}
\typeout{************************************************}
\section[Student Learning Outcomes]{Student Learning Outcomes}\label{learning-outcomes-functions}
\leavevmode%
\begin{itemize}[label=\textbullet]
\item{}Identify a function from a table of values, a graph or an equation.%
\item{}Recognize, apply, interpret, evaluate and solve equations using function notation.%
\item{}Define a relation and define a function.%
\item{}Perform calculations using function notation including Average Rate of Change.%
\item{}Determine if a variable is dependent or independent.%
\item{}Determine the intervals over which a function is increasing or decreasing or constant based on a numerical, graphical or algebraic model.%
\item{}Demonstrate appropriate use of inequality notation and interval notation.%
\end{itemize}
\typeout{************************************************}
\typeout{Section 1.2 Gist}
\typeout{************************************************}
\section[Gist]{Gist}\label{functions-gist}
A function is a rule that be in the form of a graph or a table of values or a formula.  It may even be a sentence or a set of instructions.  A function takes an input value and uses the rule to create an output value.%
\par
Function notation looks like this: \(f(\text{input})=\text{output}\)%
\par
But, instead of writing the words ``input'' and ``output'' we usually use variables, like x and y. Then we define in words what the variables actually mean or represent.  Most often the notation will look something like this: \(f(x)=y\)
%
\par
Inside the parentheses is the \terminology{Independent Variable}, like \(x\), and outside the parentheses, on the other side of the equals sign, is the \terminology{Dependent Variable}, like \(y\).%
\begin{example}\label{example-1}
Let \(E\) be the fuel efficiency, in miles per gallon, of a car traveling at \(s\) miles per hour. For speeds between 5 and 75 miles per hour, the function in \hyperref[speed-vs-efficiency]{\ref{speed-vs-efficiency}} tells us the efficiency of the car.%
\leavevmode%
\begin{figure}
\centering
{
\begin{tikzpicture}
\begin{axis}[ymin=0, ymax=35, xlabel={Speed (mph)}, ylabel={Fuel Economy (mpg)}, minor xtick={5,10,...,75}, ytick={0,10,...,30}, minor ytick={0,5,...,35}, xmin=5, xmax=75, grid=both, axis background/.style={fill=yellow!40}]
\addplot[smooth, thick, color=red] coordinates {(5, 11) (10, 19) (15, 24) (20, 26.5) (25, 28) (30,29) (35, 29.5) (40, 30) (55, 30.5) (65,26.5) (70, 25) (75, 24)};
\end{axis}
\end{tikzpicture}
}
\caption{Fuel Economy versus Speed\label{speed-vs-efficiency}}
\end{figure}
\par
Therefore, the \emph{efficiency is a function of the speed}. Using function notation we could write an equation saying the same thing: \(E=f(s)\)%
\par
The input is the speed, \(s\), and the output is the efficiency \(E\). The function \(f(s)\) is the graph itself.%
\par
The efficiency of the car is given by the height of the graph.  Therefore \(f(35) = 28\) means, ``The efficiency of the car traveling at \SI{35}{\mileperhour} is 28 mpg''.%
\end{example}
\typeout{************************************************}
\typeout{Subsection 1.2.1 Inputs have \emph{unique} outputs}
\typeout{************************************************}
\subsection[Inputs have \emph{unique} outputs]{Inputs have \emph{unique} outputs}\label{subsection-1}
An important aspect of functions is that each input can only have one output.  For instance the car can only have one efficiency at a given speed. The cannot get 10 mpg and 20 mpg simultaneously.%
\par
However, it is possible there are multiple inputs that give you the same output.  From the graph we see that a fuel efficiency of 25 mpg can be attained by traveling at a speed of about \SI{20}{\mileperhour} and also at about \SI{70}{\mileperhour}.%
\typeout{************************************************}
\typeout{Subsection 1.2.2 Describing functions with intervals}
\typeout{************************************************}
\subsection[Describing functions with intervals]{Describing functions with intervals}\label{subsection-2}
A function may have many different characteristics. The function may increase or decrease, it may curve up or curve down, it may have positive outputs or negative outputs. When we describe a function, we make our observations relative to the intervals (sections) of the input on which the observation takes place.%
\par
For instance, using our efficiency example, the efficiency may be the lowest at about \SI{5}{\mileperhour} and highest at about \SI{55}{\mileperhour}. The efficiency increases (goes up) between \num{5} and \SI{55}{\mileperhour} then decreases (goes down) for speeds between \num{55} and \SI{75}{\mileperhour}.%
\par
Any portion of the speed (horizontal) axis or the efficiency (vertical) axis, is called an interval.  An interval represents a section of the input axis where something special occurs on the function.%
\par
We can use inequalities to describe when the efficiency is increasing, \(5\lt s\lt55\).  This means that fuel efficiency increases for speeds between \num{5} and \SI{55}{\mileperhour}.%
\par
Using inequalities we can say the efficiency graph is%
\leavevmode%
\begin{itemize}[label=\textbullet]
\item{}\terminology{concave down} (bends downwards) for \(5\lt s\lt60\).%
\item{}\terminology{concave up} (bends downwards) for \(60\lt s\lt75\).%
\end{itemize}
\typeout{************************************************}
\typeout{Subsection 1.2.3 Evalueate vs. Solve}
\typeout{************************************************}
\subsection[Evalueate vs. Solve]{Evalueate vs. Solve}\label{subsection-3}
\terminology{Evaluate} means to use a known input value to find the output value. Evaluating a function looks like this:%
\[f(25)\]\par
It means, ``Find the result of choosing \SI{25}{\mileperhour}''. The symbol \(f(25)\) represents an \terminology{output}.%
\par
From the graph we see that \(f(25)\approx27\).%
\par
\terminology{Solve} means the output is already known and we are trying to find all the possible inputs that give us the desired result. Solving an equation will look like this:%
\[\text{Solve }f(s)=25\]\par
In other words it means, ``Find the input or inputs that give us an output of 25''.%
\par
Using our efficiency graph, we can solve the equation \(f(s)=25\).%
\par
At the speeds \SI{20}{\mileperhour} and \SI{70}{\mileperhour} the car will have an efficiency of 25 mpg.%
\typeout{************************************************}
\typeout{Exercises 1.2.4 Exercises}
\typeout{************************************************}
\subsection[Exercises]{Exercises}\label{exercises-1}
\begin{exerciselist}
\item[1.]\hypertarget{exercise-1}{}(Equation for a Secant Line)\space\space{}\par
\noindent%
\textbf{Problem.}\quad Find a formula for the line passing through the function \(f(x) = \frac{10}{x^2+1}\)  at \(x = 1\) and \(x = 3\).%
\par
 \framebox[20em]{\raisebox{1ex}{}}%
\par
\noindent%
\textbf{Solution.}\quad Since \(f(1) = 5\), and \(f(3) = 1\), then the slope of this line is \(\frac{1 - 5}{3 - 1}=2\). Using the point-slope form, we can write the line's equation as \(y=2(x-1)+5\).%
\par
\end{exerciselist}
\typeout{************************************************}
\typeout{Chapter 2 Domain, Range, and Piecewise-Defined Functions}
\typeout{************************************************}
\chapter[Domain, Range, and Piecewise-Defined Functions]{Domain, Range, and Piecewise-Defined Functions}\label{domain-range-piecewise}
\typeout{************************************************}
\typeout{Chapter 3 Exponential Functions}
\typeout{************************************************}
\chapter[Exponential Functions]{Exponential Functions}\label{exponential-functions}
\typeout{************************************************}
\typeout{Chapter 4 Logarithmic Functions}
\typeout{************************************************}
\chapter[Logarithmic Functions]{Logarithmic Functions}\label{logarithmic-functions}
\typeout{************************************************}
\typeout{Chapter 5 Vertical and Horizontal Translations}
\typeout{************************************************}
\chapter[Vertical and Horizontal Translations]{Vertical and Horizontal Translations}\label{vertical-and-horizontal-translations}
\typeout{************************************************}
\typeout{Chapter 6 Reflections and Vertical Stretches}
\typeout{************************************************}
\chapter[Reflections and Vertical Stretches]{Reflections and Vertical Stretches}\label{reflections-and-vertical-stretches}
\typeout{************************************************}
\typeout{Chapter 7 Composition and Inverse Functions}
\typeout{************************************************}
\chapter[Composition and Inverse Functions]{Composition and Inverse Functions}\label{composition-and-inverse}
\typeout{************************************************}
\typeout{Chapter 8 Combinations of Funcitons}
\typeout{************************************************}
\chapter[Combinations of Funcitons]{Combinations of Funcitons}\label{combinations-of-functions}
\typeout{************************************************}
\typeout{Chapter 9 Power Functions and Polynomials}
\typeout{************************************************}
\chapter[Power Functions and Polynomials]{Power Functions and Polynomials}\label{power-functions-and-polynomials}
\end{document}